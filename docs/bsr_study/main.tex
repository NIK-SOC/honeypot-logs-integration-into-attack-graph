\documentclass[conference]{IEEEtran}

\usepackage{graphicx}
\usepackage{amsmath}
\usepackage{hyperref}

\begin{document}

\title{Integration of Honeypot Logs into Neo4j Graph Database Using the STIX Data Model}

\author{\IEEEauthorblockN{1\textsuperscript{st} Sándor Róbert Bakos}
\IEEEauthorblockA{\textit{John von Neumann Faculty of Informatics} \\
\textit{Obuda University}\\
Budapest, Hungary \\
sandor.r.bakos@stud.uni-obuda.hu}
\and
\IEEEauthorblockN{2\textsuperscript{nd} Anna Bánáti}
\IEEEauthorblockA{\textit{John von Neumann Faculty of Informatics} \\
\textit{Obuda University}\\
Budapest, Hungary \\
banati.anna@nik.uni-obuda.hu}
}

\maketitle

\begin{abstract}
This paper aims to demonstrate the integration of honeypot logs into a Neo4j graph database using the STIX data model. The study focuses on creating and analyzing attack graphs for identifying vulnerabilities and attack patterns while exploring their applicability in optimizing Security Operation Centers (SOC).
\end{abstract}

\begin{IEEEkeywords}
honeypot, STIX, attack graph, Neo4j, SOC
\end{IEEEkeywords}

\thanks{This document was partially written with the assistance of ChatGPT, an AI language model developed by OpenAI.\cite{openai2024chatgpt}}

\section{Introduction}

In the rapidly evolving landscape of cybersecurity, the ability to detect, analyze, and mitigate threats is paramount for maintaining secure systems. Cyber attackers continuously develop sophisticated methods to exploit vulnerabilities, necessitating proactive defense mechanisms~\cite{mirkovic2004taxonomy}. Honeypots, which are decoy systems designed to attract attackers, generate vast amounts of data that provide valuable insights into malicious activities~\cite{spitzner2003honeypots, provos2007virtual}. By simulating vulnerable systems, honeypots capture detailed information about attack vectors, tools, and methodologies employed by adversaries. However, analyzing these logs in their raw form presents significant challenges due to their volume, complexity, and lack of standardization~\cite{katti2018honeypot}.

Traditional data analysis techniques may not efficiently handle the complex relationships inherent in cybersecurity data. Graph databases, such as Neo4j, offer powerful means to model and analyze these relationships effectively~\cite{robinson2015graph}. Furthermore, the Structured Threat Information Expression (STIX) data model provides a standardized framework for representing cyber threat intelligence, facilitating information sharing and interoperability~\cite{oasis2023stix}.

This paper aims to demonstrate the integration of honeypot logs into a Neo4j graph database using the STIX data model. By leveraging Neo4j's graph capabilities and STIX's structured approach to threat intelligence, this study seeks to create and analyze attack graphs to identify vulnerabilities, track attack paths, and derive actionable insights. This integration enables the visualization of complex attack patterns and supports more effective threat analysis and response.

The novelty of this research lies in its focus on applying a graph-based approach to honeypot log analysis within the STIX framework. While previous studies have explored honeypot data analysis and the use of graph databases separately, the combined application in this context is less explored~\cite{sharma2018utilizing}. This approach not only enhances the visualization of attack patterns but also facilitates better decision-making in Security Operations Centers (SOCs) by providing a comprehensive view of threats~\cite{ring2019survey}. Additionally, it offers a scalable and standardized method for transforming and integrating threat intelligence data, promoting collaboration across organizations~\cite{barnum2012standardizing}.

The remainder of this paper is structured as follows: \textbf{Section 2} provides background on honeypots, the STIX data model, and Neo4j. \textbf{Section 3} describes the methodology for integrating honeypot logs into a graph database. \textbf{Section 4} presents a case study to demonstrate the application of the proposed approach. \textbf{Section 5} discusses the results and implications of the study, while \textbf{Section 6} concludes with key findings and directions for future work.




\section{Background}
\subsection{Honeypot Technology}
Honeypots are designed to gather intelligence on attack methods and deceive attackers. They have been widely used in SOC environments for monitoring and profiling adversarial behavior \cite{honeypotsurvey}.

\subsection{STIX Data Model}
The Structured Threat Information Expression (STIX) data model provides a standardized approach for representing and sharing cyber threat intelligence. STIX objects, such as domain objects (e.g., attack patterns, vulnerabilities) and relationships (e.g., "targets" or "uses"), form the building blocks of structured cyber threat data \cite{simonnagy2022attackgraphs}.


\subsection{Neo4j Graph Database}
Neo4j's Labeled Property Graph (LPG) model is highly suited for representing STIX data due to its ability to handle nodes and relationships with properties efficiently. The use of Cypher, Neo4j’s query language, enables fast traversal and pattern matching, making it ideal for analyzing complex attack graphs \cite{simonnagy2022attackgraphs}.


\subsection{Attack Graphs}
Attack graphs represent vulnerabilities and potential attack paths in a system. They are commonly used in SOCs to enhance risk assessment and prioritize security measures \cite{attackgraphsurvey}.

\section{Methodology}
\subsection{Synthetic Honeypot Log Generation}

To demonstrate the integration of honeypot logs into a Neo4j graph database using the STIX data model, we developed a Python script to generate synthetic honeypot logs. This approach allowed us to create controlled datasets that mimic real-world cyber-attack patterns while avoiding the ethical and legal implications of using real attack data~\cite{provos2007virtual}.

\subsubsection{Methodology}

We designed a Python script that generates synthetic honeypot logs with the following key features:

\begin{enumerate}
    \item \textbf{Attack Simulation with Logical Dependencies}: The script simulates common cyber attacks by randomly selecting attack types such as SSH brute-force, SQL injection, and DDoS attacks. Each attack type is logically linked to appropriate target services:
    \begin{itemize}
        \item \textbf{SSH brute-force} targets services like SSH.
        \item \textbf{SQL injection} attacks target database services such as MySQL or PostgreSQL.
        \item \textbf{DDoS} attacks target services like HTTP or DNS.
    \end{itemize}
    
    \item \textbf{Consistent Payload Format}: All payloads are structured in a consistent data format (JSON-like structures) to ensure uniformity. Each payload includes:
    \begin{itemize}
        \item A \texttt{type} field indicating the nature of the payload (e.g., \texttt{credentials}, \texttt{query}, \texttt{request}).
        \item A \texttt{content} field containing the actual data or commands used in the attack.
    \end{itemize}
    
    \item \textbf{Realistic Sample Payloads}: The script incorporates realistic payloads for each attack type:
    \begin{itemize}
        \item \textbf{SSH brute-force}: Simulated username and password combinations commonly used in brute-force attempts.
        \item \textbf{SQL injection}: Malicious SQL queries designed to exploit database vulnerabilities.
        \item \textbf{DDoS}: Simulated high-traffic requests or packets aimed at overwhelming target services.
    \end{itemize}
    
    \item \textbf{Random Data Generation}: Each log entry includes randomized details to mimic diverse attack scenarios:
    \begin{itemize}
        \item \textbf{Source IP Addresses}: Randomly generated to represent different attack origins.
        \item \textbf{Timestamps}: Randomized within a recent time frame to reflect realistic attack timings.
        \item \textbf{Target Services}: Selected based on logical dependencies with the attack type.
    \end{itemize}
    
    \item \textbf{Data Storage}: The generated logs are saved in a JSON file format.
\end{enumerate}

\subsubsection{Tools and Technologies}

\begin{itemize}
    \item \textbf{Python}: Used for scripting the data generation process.
    \item \textbf{JSON}: Utilized for storing the synthetic logs in a structured format.
\end{itemize}

\subsubsection{Ethical Considerations}

By using synthetic data, we avoided issues related to privacy, data sensitivity, and legal restrictions~\cite{acm2018ethics}. This approach ensured compliance with ethical standards in cybersecurity research and eliminated risks associated with handling real attack data.




\subsection{Transforming Logs into STIX Format}
Logs are converted into STIX objects (e.g., attack patterns, indicators) and relationships to represent attack scenarios.

\subsection{Loading Data into Neo4j}
The STIX data is imported into Neo4j, creating a graph representation of the attack scenarios.

\subsection{Graph Visualization and Analysis}
Attack graphs are visualized using Neo4j's query and visualization tools, enabling insights into attack paths and vulnerabilities.

\section{Case Study}
A honeypot log dataset is used to demonstrate the process. The resulting attack graph highlights critical vulnerabilities and paths, as illustrated in related works \cite{banati2022attackgraphs}.

\section{Results and Discussion}
\subsection{Advantages of Graph-Based Analysis}
Graph-based methods enable efficient visualization and prioritization of security risks, aligning with studies like \cite{banati2022attackgraphs}.

\subsection{Challenges}
Scalability and dynamic updates of attack graphs remain significant challenges, as noted in \cite{simonnagy2022attackgraphs}. Addressing these issues requires optimized graph traversal algorithms and real-time data integration.


\subsection{Opportunities}
Future work could focus on real-time graph updates and dynamic risk assessment methods, as proposed by \cite{neo4jguide}.

\section{Conclusion and Future Work}
This study demonstrates the integration of honeypot logs into Neo4j using the STIX data model, emphasizing its utility for SOC optimization. Future work will focus on real-time graph models and scalability improvements.

\section*{Acknowledgment}
The authors thank [Name/Institution] for supporting this research.

\bibliographystyle{IEEEtran}
\bibliography{references}

\end{document}
